\chapter{Introduction} % Introduction
\label{ch:intro}

\section{Motivation}

In the vast and developing world of algorithms and data structures, one could find quite a lot of different topics. 

The Graphs could be considered as one of the most complex of them all! You might find it challenging when you first encounter the concept and theory behind them. ``Graphs are one of the unifying themes of computer science—an abstract representation that describes the organization of transportation systems, human interactions, and telecommunication networks. That so many different structures can be modeled using a single formalism is a source of great power to the educated programmer. More precisely, a graph G=(V,E) consists of a set of vertices V together with a set E of vertex pairs or edges.``~\cite{skiena-algorithm-design}. 

We truly grow, when we face the challenges and seize the opportunity, no matter how hard it may seem to achieve! So the true motivation behind this very thesis is to dive into the difficult world of graphs myself and to guide others by showing how exactly certain things work with graphs. And I do such, by visualizing the graph and the algorithm being applied to the graph. It is always great to have a tool by your hand, which could show you how exactly certain parts might look like in practice when you learn the theory. 

\section{Thesis goal}

So abstract yet, so practical in the real world! Different applications of graph data structure is wide-spread around the world: in networking schema, circuits, maps and etc. With all of the aforementioned applications, we might encounter many kinds of challenges and obstacles, one of which, is what my whole thesis is about - Path Finding. Let's say you have a certain node A and the other node B, having that all of the nodes in graph are connected to each other with edges, we would like to find a path from our node A to node B through those connections. 

That problem carries a name - Pathfinding or as Jon and Éva call it in their book, s-t connectivity. ``Suppose we are given a graph G=(V,E) and two particular nodes s and t. We’d like to find an efficient algorithm that answers the question: Is there a path from s to t in G? We will call this the problem of determining s-t connectivity.``~\cite{jon-and-eva-algorithm-design}

Something so simple and intuitive to the human being is not an easy goal for the program to achieve. Traversal of the elementary unweighted graph would not be of trouble, but what if you have a massive network consisting of thousands nodes and edges, where every edge has some weight. A trivial traversal through all of the possible paths would be terribly ineffective in this case. Pathfinding algorithms reach the goal in the most efficient ways, so it is crucial to know them for everyone who is going to deep dive into the world of graphs. 

The main goal of the thesis is to help people understand the ways algorithms traverse through the graph by visualizing them. Handing to the learner the tool which might help them to better understand what really happens when the algorithm is being executed.