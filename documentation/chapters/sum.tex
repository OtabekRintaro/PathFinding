\chapter{Conclusion} % Conclusion
\label{ch:sum}

Summing up, the documentation have covered such topics as graphs, algorithms on them and how one could implement the visualization on them. These graphs are truly a fascinating field to learn, as it brings further understanding in various other areas of life, where one could not have seen any evidence of graph application. 
The visualization helps people to understand certain topics way easier. At times, it could even wake a curiosity in a man.

Throughout the thesis, there had been many challenges met, which concluded in certain decisions during its development. Some of them were avoided thanks to the proper planning phase, making the components less coupled, others were indicated early in the process and had been solved swiftly as the application of TDD made the development more robust and gradual.

Even though the tool has satisfied its requirements, it is not yet perfect. As one of the big improvements in the future, better control over the algorithm and description of the every step could be implemented. The users might encounter certain difficulties when the visualization tool is used alone, rather it is more effective with the application of theoretical knowledge that one could get by reading relevant books. Therefore, adding more of a description behind the algorithms decisions and steps would ease the process of learning for the user.

During the process, I have learned about such libraries as Flask and React, latter of which had concepts truly hard to grasp. Also, application of my knowledge on graphs helped to see them from the different perspective as well as to achieving comprehension of the algorithms that have been implemented.